\chapter{Preliminaries: Set theory and categories}
\section{Naive Set Theory}\label{sec:1.1}

\subsection{Sets}\label{sec:1.1.1}

一个 set 由它所包含无序的、无重复的 \emph{elements} 决定。set 的一个变种是 `multiset', 它的 elements 允许重复, 正确定义一个 multiset 的方式是使用 \emph{functions}。

一些有名的 sets:
\begin{itemize}
    \item \(\emptyset\):空集;
    \item \(\mathbb{N}\):自然数集;
    \item \(\mathbb{Z}\):整数集;
    \item \(\mathbb{Q}\):有理数集;
    \item \(\mathbb{R}\):实数集;
    \item \(\mathbb{C}\):复数集。
\end{itemize}

另外, 用 \emph{singleton} 表示只有一个元素的 set。

下面是一些有用的符号(称为 \emph{quantifiers}):

\begin{itemize}
    \item \(\exists\): 存在;
    \item \(\forall\): 对所有;
    \item \(\exists!\): 存在唯一。
\end{itemize}

\subsection{Inclusion of sets}\label{sec:1.1.2}

我们用 \(S \subseteq T\) 来表示 \(S\) 是 \(T\) 的子集。

\subsection{Operations between sets}\label{sec:1.1.3}

\begin{itemize}
    \item \(\cup\): 并集;
    \item \(\cap\): 交集;
    \item \(\setminus\): 差集;
    \item \(\coprod\): 对称差;
    \item \(\times\): 笛卡尔积;
    \item `quotient by an equivalence relation'.
\end{itemize}

\subsection{Equivalence relations, partitions, quotients}\label{sec:1.1.4}

\begin{definition}[Equivalence relation]\label{def:equivalence_relation}
    An \emph{equivalence relation} on a set \(S\) is a relation \(\sim\) satisfying the following properties:
    \begin{itemize}
        \item \emph{Reflexivity}: \(a \sim a\) for all \(a \in S\);
        \item \emph{Symmetry}: if \(a \sim b\), then \(b \sim a\);
        \item \emph{Transitivity}: if \(a \sim b\) and \(b \sim c\), then \(a \sim c\).
    \end{itemize}
\end{definition}

\begin{definition}[Quotient]\label{def:quotient}
    The \emph{quotient} of a set \(S\) with respect to an equivalence relation \(\sim\) is the set
    \[
        S / \sim := \mathscr{P}_{\sim }
    \]
    of equivalence classes of elements of \(S\) with respect to \(\sim\).
\end{definition}
\begin{eg}\label{eg:1.1.3}
    令 \(S = \mathbb{Z}\), 并定义关系 \(\sim \) 为
    \[
        a \sim b \iff a - b \text{ is even}.
    \]
    于是 \(\mathbb{Z} / \sim \) 由两个等价类组成:
    \[\mathbb{Z} / \sim  = \left\{[0]_{\sim }, [1]_{\sim } \right\}\]
\end{eg}

\section{Functions between sets}\label{sec:1.2}

\subsection{Definition}\label{sec:1.2.1}

sets 之间使用 functions 进行交互, function \(f : A \to B\) 的所有信息就是 \emph{which element \(b\) of \(B\) is the image of any given element of \(a\) of \(A\)}, 这些信息相当于一些二元组的集合, 即:
\[\Gamma_f := \left\{ \left(a, b\right) \in A \times B \mid b = f(a) \right\} \subseteq A \times B \]
而集合 \(\Gamma_f\) 通常被称为 \(f\) 的 \emph{graph}。

为了满足函数的定义, 集合 \(\Gamma_f\) 需要满足下列约束:
\[\left(\forall a \in A\right)\left(\exists! b \in B\right) \quad \left(a, b\right) \in \Gamma_f\]

我们声明 \(f\) 是从 \(A\) 到 \(B\) 的函数, 通常记作 \(f : A \to B\), 也可以画成下图(`diagram')的形式:
% https://q.uiver.app/#q=WzAsMixbMCwwLCJBIl0sWzEsMCwiQiJdLFswLDEsImYiXV0=
\[\begin{tikzcd}
        A & B
        \arrow["f", from=1-1, to=1-2]
    \end{tikzcd}\]
对于任意 \(a \in A\), 我们记作:
\[a \mapsto f(a).\]

如果 \(S\) 是 \(A\) 的子集, 我们可以定义 \(f(S)\) 为:
\[f(S) := \left\{ b \in B \mid \left( \exists a \in A\right) b = f(a)\right\}.\]
也就是说, \(f(S)\) 是包含所有 \(S\) 中的 elements 的像的集合。最大的这样的集合是 \(f(A)\), 也记为 \(\im f\)。

另外, 我们用 \(f\mid_S\) 表示将 \(f\) 限制在 \(S\) 上的函数:即 function \(S \to B\) defined by
\[\left( \forall s \in S\right): f\mid_S (s) = f(s).\]
\begin{note}
    \(f(S) = \im(f\mid_S).\)
\end{note}

\subsection{Examples: Multisetsm, indexed sets}\label{sec:1.2.2}

\begin{eg}[Multiset]
    我们用一个 function 来定义 multiset, 一种可能的方法是使用一个 function \(f : A \to \mathbb{N}^{*}\), 其中 \(A\) 是一个集合。这样, \(f(a)\) 就是 \(a\) 在 multiset 中的重复次数。
\end{eg}
\begin{eg}[Indexed set]
    同样使用 function 来得到一个 indexed set。将集合 \(I\) 看作下标, 我们可以用一个 function \(f : I \to A\) 来给 \(\im(I) \subseteq A\) 中的每个元素一个序号, 从而得到一个 indexed set.
\end{eg}

\subsection{Composition of functions}\label{sec:1.2.3}

functions 可以被组合, 即给定两个 functions \(f : A \to B\) 和 \(g : B \to C\), 我们可以定义它们的 \emph{composition} \(gf : A \to C\) 为:
\[(\forall a \in A) \quad (g \circ f)(a) := g(f(a))\]

用 diagram 的形式表示为:
% https://q.uiver.app/#q=WzAsMyxbMCwwLCJBIl0sWzEsMCwiQiJdLFsyLDAsIkMiXSxbMCwxLCJmIl0sWzEsMiwiZyJdLFswLDIsImcgXFxjaXJjIGYiLDIseyJjdXJ2ZSI6Mn1dXQ==
\[\begin{tikzcd}
        A & B & C
        \arrow["f", from=1-1, to=1-2]
        \arrow["{g \circ f}"', curve={height=12pt}, from=1-1, to=1-3]
        \arrow["g", from=1-2, to=1-3]
    \end{tikzcd}\]

我们称上图 \emph{commutes} 或是 \emph{commutative} 的, 表示无论 \(A\) 从那条路径到 \(C\), 得到的结果都是一样的。

\begin{note}[Associativity]
    composition 是 \emph{associative} 的, 即对于任意三个 functions \(f : A \to B, g : B \to C, h : C \to D\), 有 \(h \circ (g \circ f) = (h \circ g) \circ f\). 在 diagram 中表示为:
    % https://q.uiver.app/#q=WzAsNCxbMCwwLCJBIl0sWzEsMCwiQiJdLFsyLDAsIkMiXSxbMywwLCJEIl0sWzAsMSwiZiJdLFsxLDIsImciXSxbMCwyLCJnIFxcY2lyYyBmIiwyLHsiY3VydmUiOjJ9XSxbMiwzLCJoIl0sWzEsMywiaCBcXGNpcmMgZyIsMCx7Im9mZnNldCI6LTEsImN1cnZlIjotMn1dXQ==
    \[\begin{tikzcd}
            A & B & C & D
            \arrow["f", from=1-1, to=1-2]
            \arrow["{g \circ f}"', curve={height=12pt}, from=1-1, to=1-3]
            \arrow["g", from=1-2, to=1-3]
            \arrow["{h \circ g}", shift left, curve={height=-12pt}, from=1-2, to=1-4]
            \arrow["h", from=1-3, to=1-4]
        \end{tikzcd}\]
\end{note}

\subsection{Injections, surjections, bijections}\label{sec:1.2.4}

\begin{itemize}
    \item \emph{Injection}: \(f : A \to B\) is an \emph{injection} if for all \(a, a^{\prime} \in A\), \(f(a) = f(a^{\prime})\) implies \(a = a^{\prime}\).
    \item \emph{Surjection}: \(f : A \to B\) is a \emph{surjection} if for all \(b \in B\), there exists \(a \in A\) such that \(f(a) = b\).
    \item \emph{Bijection}: \(f : A \to B\) is a \emph{bijection} if it is both an injection and a surjection.
\end{itemize}

injections 通常画作 \(\hookrightarrow\), surjections 通常画作 \(\twoheadrightarrow\).

如果 \(f\) 既是 injection 又是 surjection, 我们称 \(f\) 是 \emph{bijective} 的(或称之为一个 \emph{bijection} 或 \emph{one to one correspondence} 或 \emph{isomorphism of sets}), 写作 \(f : A \xrightarrow{\sim} B\) 或
\[A \cong B,\]
此时我们称 \(A\) 和 \(B\) 是 `\emph{isomorphic}' sets.

\subsection{Injections, surjections, bijections: Second viewpoint}\label{sec:1.2.5}

回顾前面对 \(\Gamma_f\) 的约束, 我们希望定义一个函数 \(g : B \to A\), 使得 \(\forall a \in A, \exists b \in B, s.t. f(a) = b \implies g(b) = a\), 当 \(f\) 是 surjection 时, \(\Gamma_g\) 满足约束中的 \(\forall b \in B\);当 \(f\) 是 injection 时, \(\Gamma_g\) 满足约束中的 \(\exists! a \in A\)。也就是说, 当 \(f\) 是 bijection 时, 存在唯一的函数 \(g\) 是 \(f\) 的逆函数。

从图形上看, \(g\) 有一个有趣的性质, 即下面的图形是 commutative 的:
% https://q.uiver.app/#q=WzAsNyxbMCwwLCJBIl0sWzEsMCwiQiJdLFsyLDAsIkEiXSxbMywwLCLvvIwiXSxbNCwwLCJCIl0sWzUsMCwiQSJdLFs2LDAsIkIiXSxbMCwxLCJmIl0sWzEsMiwiZyJdLFswLDIsIlxcaWRfQSIsMix7ImN1cnZlIjoyfV0sWzQsNSwiZyJdLFs1LDYsImYiXSxbNCw2LCJcXGlkX0IiLDIseyJjdXJ2ZSI6Mn1dXQ==
\[\begin{tikzcd}
        A & B & A & {, } & B & A & B
        \arrow["f", from=1-1, to=1-2]
        \arrow["{\id_A}"', curve={height=12pt}, from=1-1, to=1-3]
        \arrow["g", from=1-2, to=1-3]
        \arrow["g", from=1-5, to=1-6]
        \arrow["{\id_B}"', curve={height=12pt}, from=1-5, to=1-7]
        \arrow["f", from=1-6, to=1-7]
    \end{tikzcd}\]

\begin{proposition}
    Assume \(A \not= \emptyset\), and let \(f : A \to B\) be a function. Then
    \begin{enumerate}
        \item \(f\) has a left-inverse if and only if \(f\) is injective.
        \item \(f\) has a right-inverse if and only if \(f\) is surjective.
    \end{enumerate}
\end{proposition}

\begin{corollary}
    A function \(f : A \to B\) is a bijection if and only if it has a (two-sided) inverse.
\end{corollary}

\subsection{Monomorphisms and epimorphisms}\label{sec:1.2.6}

monomorphisms 和 epimorphisms 是 category theory 中的概念, 可以用更图形化的视角来看待 injectivity 和 surjectivity.

A function $f: A \to B$ is called a \emph{monomorphism(monic)} if for any sets $Z$ and functions $\alpha^{\prime}, \alpha^{\prime\prime}: Z \to A$, $f \circ \alpha^{\prime} = f \alpha^{\prime\prime}$ implies $\alpha^{\prime}= \alpha^{\prime\prime}$.

\begin{proposition}
    A function is injective if and only if it is a monomorphism.
\end{proposition}

类似的, epimorphisms 则对应 surjectivity.

\subsection{Basic examples}\label{sec:1.2.7}

\begin{eg}[Natural projections]\label{eg:1.2.4}
    令 \(A, B\) 为非空集合, 则 \emph{natural projections} \(\pi_A, \pi_B\):
    % https://q.uiver.app/#q=WzAsMyxbMSwwLCJBIFxcdGltZXMgQiJdLFswLDEsIkEiXSxbMiwxLCJCIl0sWzAsMSwiXFxwaV9BIiwyLHsic3R5bGUiOnsiaGVhZCI6eyJuYW1lIjoiZXBpIn19fV0sWzAsMiwiXFxwaV9CIiwwLHsic3R5bGUiOnsiaGVhZCI6eyJuYW1lIjoiZXBpIn19fV1d
    \[\begin{tikzcd}
            & {A \times B} \\
            A && B
            \arrow["{\pi_A}"', two heads, from=1-2, to=2-1]
            \arrow["{\pi_B}", two heads, from=1-2, to=2-3]
        \end{tikzcd}\]
    被定义为 \(\pi_A(a, b) := a, \pi_B(a, b) := b, \forall (a, b)\in A\times B\)。这两个映射都是 surjections.
\end{eg}
\begin{eg}[Natural injections]\label{eg:1.2.5}
    类似的, 有 natural injections from \(A\) and \(B\) to the disjoint union \(A \coprod B\):
    % https://q.uiver.app/#q=WzAsMyxbMCwwLCJBIl0sWzIsMCwiQiJdLFsxLDEsIkEgXFxzcWN1cCBCIl0sWzAsMiwiIiwwLHsic3R5bGUiOnsidGFpbCI6eyJuYW1lIjoiaG9vayIsInNpZGUiOiJ0b3AifX19XSxbMSwyLCIiLDIseyJzdHlsZSI6eyJ0YWlsIjp7Im5hbWUiOiJob29rIiwic2lkZSI6ImJvdHRvbSJ9fX1dXQ==
    \[\begin{tikzcd}
            A && B \\
            & {A \coprod B}
            \arrow[hook, from=1-1, to=2-2]
            \arrow[hook', from=1-3, to=2-2]
        \end{tikzcd}\]
    即将 \(a \in A\) 映射到 \((a, 0) \in A \coprod B\), 将 \(b \in B\) 映射到 \((b, 1) \in A \coprod B\).
\end{eg}
\begin{eg}[Canonical projection]\label{eg:1.2.6}
    假设 \(\sim \) 是集合 \(A\) 上的等价关系, 于是有(surjective)canonical projection:
    % https://q.uiver.app/#q=WzAsMixbMCwwLCJBIl0sWzEsMCwiQSAvIFxcc2ltIl0sWzAsMSwiIiwyLHsic3R5bGUiOnsiaGVhZCI6eyJuYW1lIjoiZXBpIn19fV1d
    \[\begin{tikzcd}
            A & {A / \sim}
            \arrow[two heads, from=1-1, to=1-2]
        \end{tikzcd}\]
    即将 \(a \in A\) 映射到它的等价类 \([a]_{\sim }\).
\end{eg}

\subsection{Canonical decomposition}\label{sec:1.2.8}
We define an equivalence relation $\sim$ on set $A$ as follows: for all $a^{\prime}, a^{\prime\prime}\in A, a'\sim a''\iff f(a')=f(a'')$.

\begin{theorem}[Canonical decomposition]\label{thm:1.2.7}
    Let $f: A \to B$ be any function, and define $\sim$ as above. Then $f$ decomposes as follows:
    % https://q.uiver.app/#q=WzAsNCxbMCwwLCJBIl0sWzEsMCwiKEEgLyBcXHNpbSkiXSxbMiwwLCJcXGltIGYiXSxbMywwLCJCIl0sWzAsMSwiIiwyLHsic3R5bGUiOnsiaGVhZCI6eyJuYW1lIjoiZXBpIn19fV0sWzEsMiwiXFx0aWxkZXtmfSIsMl0sWzIsMywiIiwyLHsic3R5bGUiOnsidGFpbCI6eyJuYW1lIjoiaG9vayIsInNpZGUiOiJ0b3AifX19XSxbMCwzLCJmIiwwLHsiY3VydmUiOi0zfV1d
    \[\begin{tikzcd}
            A & {(A / \sim)} & {\im f} & B
            \arrow[two heads, from=1-1, to=1-2]
            \arrow["f", curve={height=-18pt}, from=1-1, to=1-4]
            \arrow["{\sim}", "{\tilde{f}}"', from=1-2, to=1-3]
            \arrow[hook, from=1-3, to=1-4]
        \end{tikzcd}\]
    where the first function is the canonical projection $A \to A/\sim$, the third function is the inclusion induced $\operatorname{im}f \subseteq B$, and the bijection $\tilde{f}$ in the middle is defined by
    $$\tilde{f}([a]_{\sim}) \coloneqq f(a)$$
    for all $a \in A$.
\end{theorem}
\begin{remark}
    可以看出, 任何一个函数都可以分解为一个 surjection, 一个 bijection 和一个 injection 的组合, 这个定理的 key insight 在于按照 \(\im f\) 将 \(A\) 拆分为等价类。
\end{remark}

如果对 \(\tilde{f}\) 的定义合法, 那么显然上述分解是 commutative 的。因此证明这个定理的关键在于:
\begin{itemize}
    \item \(\tilde{f}\) 确实是一个 function;
    \item \(\tilde{f}\) 确实是一个 bijection.
\end{itemize}

在上述定理中对 \(\tilde{f}\) 的定义中存在一个问题, 那就是对于同一等价类中的 \(a'\) 和 \(a''\), 是否有 \(f(a') = f(a'')\), 如果是否定的话, 那么 \([a]_{\sim}\) 会对应多个 image, 从而 \(\tilde{f}\) 不是一个 function. 从前面对 \(\sim\) 的定义中, 显然 \(f(a') = f(a'')\), 因此 \(\tilde{f}\) 是 \emph{well-defined} 的。

而对于 bijection, 我们只需证明 \(\tilde{f}\) 既是 injection 又是 surjection 即可, 这里不再赘述。

\subsection{Clarification}\label{sec:1.2.9}

通过 disjoint unions, products 或 quotients 得到的集合, 它们的主要性质并不是指它们包含了什么 elements, 而是它们与其他集合之间的关系。这在之后的 Universal properties section 中会有所体现。

\section{Categories}\label{sec:1.3}

categories 的语言被 Norman Steenrod 称为 \emph{abstract nonsense}, 因为 categories 的主要研究重点是 `structure' 而非 `meaning', 这强调了你应该把注意力放在 how that set may sit in relationship with all other sets 而不是 how you run into a specific set.

\subsection{Definition}\label{sec:1.3.1}

一个 category 由一些满足特定条件的 `objects' 和 `morphisms' 组成(collection)。注意这里使用的是 `collection' 而不是 `set', 这是因为根据 Russell's paradox, 我们应该避免讨论 `set of all sets', 而 category 的 objects 是一个比 set 更广泛的概念, 它允许 `collection of all sets'.

用 technical name 来描述上述 collection 则是 \emph{class}, 即 `class of sets', 这是一个比 set 更大的概念。另一种描述是用 \emph{universe} 来定义一个足够大的 set, 然后默认所有的 categories 的所有 objects 都取自这个 universe.

在后面的一些 examples 中, 我们会看到一些 category 的 objects 是 sets, 这种 category 我们称它是 \emph{small} 的。

\begin{definition}[Category]\label{def:category}
    A \emph{category} $\Cat$ consists of
    \begin{outline}
        \1 a class $\Obj(\Cat)$ of \emph{objects} of the category; and
        \1 for every two objects $A, B$ of $\Cat$, a set $\Hom_{\Cat}(A, B)$ of \emph{morphisms} with the properties listed below.
        \2 For every object of $A$ of $\Cat$, there exists (at least) one morphism $1_A \in \Hom_{\Cat}(A, A)$, the `identity' on $A$.
        \2 One can compose morphisms: two morphisms $f \in \Hom_{\Cat}(A, B)$ and $g \in \Hom_{\Cat}(B, C)$ determine a morphism $gf \in \Hom_{\Cat}(A, C)$.
        \2 This `composition law' is associative: if \(f \in \Hom_{\Cat}(A, B), g \in \Hom_{\Cat}(B, C)\), and \(h \in \Hom_{\Cat}(C, D)\), then
        \[
            h(gf) = (hg)f.
        \]
        \2 The identity morphisms are identities with respect to composition: that is, for all \(f \in \Hom_{\Cat}(A, B)\) we have
        \[
            f 1_A = f, \quad 1_B f = f.
        \]
    \end{outline}
\end{definition}

One further requirment is that the sets
\[
    \Hom_{\Cat}(A, B), \quad \Hom_{\Cat}(C, D)
\]
be \emph{disjoint} unless \(A = C, B = D\).

\begin{note}
    值得注意的是, 上述定义中 90\% 描述的是 morphisms 的性质, 这与上面提到的 ``categories 的研究重点是 `structure' '' 的观点不谋而合, 因为 morphisms 才是描述 `structure' 的关键。

    但是, 我们也会不可避免地从 objects 的角度来看待 categories, 因为 morphisms 最终还是由 objects 决定的。
\end{note}

\begin{definition}[Endomorphism]\label{def:endomorphism}
    A morphism of an object \(A\) of a category \(\Cat\) to itself is called an \emph{endomorphism}; \(\Hom_{\Cat}(A, A)\) is denoted \(\End_{\Cat}(A)\).
    \begin{note}
        The composition defines an `operation' on \(\End_{\Cat}(A)\): if \(f, g\) are elements of \(\End_{\Cat}(A)\), so is their composition \(gf\).
    \end{note}
\end{definition}

\subsection{Examples}\label{sec:1.3.2}

\begin{eg}[Set]\label{eg:1.3.2}
    We will use \(\Set\) to denote the category of sets. Thus
    \begin{itemize}
        \item \(\Obj(\Set)\) = the class of all sets;
        \item for \(A, B\) in \(\Obj(\Set)\), \(\Hom_{\Set}(A, B) = B^A\)
    \end{itemize}

    \begin{note}
        尽管 \cref{sec:1.1.3,sec:1.1.4} 中提到了 sets 的 operations 体现了其他 categories 不一定有的 \(\Set\) 的一些有意思的 features, 但是它们与 \(\Set\) 的定义无关。
    \end{note}
\end{eg}

\begin{eg}[Slice category]\label{eg:1.3.5}
    设 \(\Cat\) 是一个 category, \(A\) 为 \(\Cat\) 中的一个 object, 我们现在由 \(A\) 构造一个 category \(\Cat_A\):
    \begin{itemize}
        \item \(\Obj(\Cat_A) = \text{all morphisms from any object of } \Cat \text{ to } A\);
        \item 对于 objects \(f_1 \in \Hom_{\Cat}(Z_1, A), f_2 \in \Hom_{\Cat}(Z_2, A)\), 定义 \(\Hom_{\Cat_A}(f_1, f_2) = \Hom_{\Cat}(Z_1, Z_2)\), 即
              % https://q.uiver.app/#q=WzAsMyxbMSwxLCJBIl0sWzAsMCwiWl8xIl0sWzIsMCwiWl8yIl0sWzEsMiwiXFxzaWdtYSJdLFsxLDAsImZfMSIsMl0sWzIsMCwiZl8yIl1d
              \[\begin{tikzcd}
                      {Z_1} && {Z_2} \\
                      & A
                      \arrow["\sigma", from=1-1, to=1-3]
                      \arrow["{f_1}"', from=1-1, to=2-2]
                      \arrow["{f_2}", from=1-3, to=2-2]
                  \end{tikzcd}\]
    \end{itemize}
    \begin{definition}[Slice category]\label{def:slice_category}
        使用这种方式构造的 category 被称为 \emph{slice category}, 它们是 \emph{comma categories} 的特例。
    \end{definition}
\end{eg}

\begin{eg}[Coslice category]\label{eg:1.3.7}
    将 \ref{eg:1.3.5}[Eg3.5] 中 objects 的箭头反过来, 类似的方法得到它的 morphisms, 这种 category 被称为 \emph{coslice category}.
    % https://q.uiver.app/#q=WzAsMyxbMSwxLCJBIl0sWzAsMCwiWl8xIl0sWzIsMCwiWl8yIl0sWzEsMCwiZl8xIiwyXSxbMiwwLCJmXzIiXSxbMSwyLCJcXHNpZ21hIl1d
    \[\begin{tikzcd}
            {Z_1} && {Z_2} \\
            & A
            \arrow["\sigma", from=1-1, to=1-3]
            \arrow["{f_1}"', from=1-1, to=2-2]
            \arrow["{f_2}", from=1-3, to=2-2]
        \end{tikzcd}\]
\end{eg}

\begin{eg}\label{eg:1.3.9}
    在 \ref{eg:1.3.5}[Eg3.5] 中, 我们固定了 \(\Cat\) 中的一个 object \(A\), 现在我们固定两个 objects \(A, B\) 来构建 category \(\Cat_{A, B}\):
    \begin{itemize}
        \item \(\Obj(\Cat_{A, B}) = \) diagrams
              % https://q.uiver.app/#q=WzAsMyxbMCwxLCJaIl0sWzEsMCwiQSJdLFsxLDIsIkIiXSxbMCwxLCJmIl0sWzAsMiwiZyIsMl1d
              \[\begin{tikzcd}
                      & A \\
                      Z \\
                      & B
                      \arrow["f", from=2-1, to=1-2]
                      \arrow["g"', from=2-1, to=3-2]
                  \end{tikzcd}\]
              in \(\Cat\)
        \item morphisms 是 \emph{commutative} diagrams
              % https://q.uiver.app/#q=WzAsNCxbMCwxLCJaXzEiXSxbMSwxLCJaXzIiXSxbMiwwLCJBIl0sWzIsMiwiQiJdLFsxLDIsImZfMiJdLFsxLDMsImdfMiIsMl0sWzAsMSwiXFxzaWdtYSJdLFswLDIsImZfMSIsMCx7ImN1cnZlIjotMX1dLFswLDMsImdfMSIsMix7ImN1cnZlIjoyfV1d
              \[\begin{tikzcd}
                      && A \\
                      {Z_1} & {Z_2} \\
                      && B
                      \arrow["{f_1}", curve={height=-6pt}, from=2-1, to=1-3]
                      \arrow["\sigma", from=2-1, to=2-2]
                      \arrow["{g_1}"', curve={height=12pt}, from=2-1, to=3-3]
                      \arrow["{f_2}", from=2-2, to=1-3]
                      \arrow["{g_2}"', from=2-2, to=3-3]
                  \end{tikzcd}\]
    \end{itemize}
    \begin{note}
        类似的, 我们也有 coslice 版 \(\Cat^{A, B}\):
        % https://q.uiver.app/#q=WzAsNCxbMCwxLCJaXzEiXSxbMSwxLCJaXzIiXSxbMiwwLCJBIl0sWzIsMiwiQiJdLFsyLDEsImZfMiIsMl0sWzMsMSwiZ18yIl0sWzEsMCwiXFxzaWdtYSIsMl0sWzIsMCwiZl8xIiwyLHsiY3VydmUiOjF9XSxbMywwLCJnXzEiLDAseyJjdXJ2ZSI6LTJ9XV0=
        \[\begin{tikzcd}
                && A \\
                {Z_1} & {Z_2} \\
                && B
                \arrow["{f_1}"', curve={height=6pt}, from=1-3, to=2-1]
                \arrow["{f_2}"', from=1-3, to=2-2]
                \arrow["\sigma"', from=2-2, to=2-1]
                \arrow["{g_1}", curve={height=-12pt}, from=3-3, to=2-1]
                \arrow["{g_2}", from=3-3, to=2-2]
            \end{tikzcd}\]
    \end{note}
\end{eg}

\begin{eg}\label{eg:1.3.10}
    作为 \(\Cat_{A, B}\) 的变种, 我们也可以考虑固定两个 \(\Cat\) 中的 morphisms \(\alpha \in \Hom_{\Cat}(A, C), \beta \in \Hom_{\Cat}(B, C)\) 得到一个 \emph{fibered} version of \(\Cat_{A, B}\):
    \begin{itemize}
        \item \(\Obj(\Cat_{\alpha, \beta}) = \) \emph{commutative diagrams}
              % https://q.uiver.app/#q=WzAsNCxbMCwxLCJaIl0sWzEsMCwiQSJdLFsxLDIsIkIiXSxbMiwxLCJDIl0sWzEsMywiXFxhbHBoYSJdLFsyLDMsIlxcYmV0YSIsMl0sWzAsMSwiZiJdLFswLDIsImciLDJdXQ==
              \[\begin{tikzcd}
                      & A \\
                      Z && C \\
                      & B
                      \arrow["\alpha", from=1-2, to=2-3]
                      \arrow["f", from=2-1, to=1-2]
                      \arrow["g"', from=2-1, to=3-2]
                      \arrow["\beta"', from=3-2, to=2-3]
                  \end{tikzcd}\] in \(\Cat\)
        \item  morphism 是响应的 \emph{commutative diagrams}
              % https://q.uiver.app/#q=WzAsNSxbMSwxLCJaXzEiXSxbMiwwLCJBIl0sWzIsMiwiQiJdLFszLDEsIkMiXSxbMCwxLCJaXzIiXSxbMSwzLCJcXGFscGhhIl0sWzIsMywiXFxiZXRhIiwyXSxbMCwxLCJmXzEiXSxbMCwyLCJnXzEiLDJdLFs0LDEsImZfMiIsMCx7ImN1cnZlIjotMX1dLFs0LDIsImdfMiIsMix7ImN1cnZlIjoyfV0sWzQsMCwiXFxzaWdtYSJdXQ==
              \[\begin{tikzcd}
                      && A \\
                      {Z_2} & {Z_1} && C \\
                      && B
                      \arrow["\alpha", from=1-3, to=2-4]
                      \arrow["{f_2}", curve={height=-6pt}, from=2-1, to=1-3]
                      \arrow["\sigma", from=2-1, to=2-2]
                      \arrow["{g_2}"', curve={height=12pt}, from=2-1, to=3-3]
                      \arrow["{f_1}", from=2-2, to=1-3]
                      \arrow["{g_1}"', from=2-2, to=3-3]
                      \arrow["\beta"', from=3-3, to=2-4]
                  \end{tikzcd}\]
    \end{itemize}
    \begin{note}
        对应的 coslice category 为 \(\Cat^{\alpha, \beta}\):
        % https://q.uiver.app/#q=WzAsNSxbMSwxLCJaXzEiXSxbMiwwLCJBIl0sWzIsMiwiQiJdLFszLDEsIkMiXSxbMCwxLCJaXzIiXSxbMywxLCJcXGFscGhhIiwyXSxbMywyLCJcXGJldGEiXSxbMSwwLCJmXzEiLDJdLFsyLDAsImdfMSJdLFsxLDQsImZfMiIsMix7ImN1cnZlIjoxfV0sWzIsNCwiZ18yIiwwLHsiY3VydmUiOi0yfV0sWzAsNCwiXFxzaWdtYSIsMl1d
        \[\begin{tikzcd}
                && A \\
                {Z_2} & {Z_1} && C \\
                && B
                \arrow["{f_2}"', curve={height=6pt}, from=1-3, to=2-1]
                \arrow["{f_1}"', from=1-3, to=2-2]
                \arrow["\sigma"', from=2-2, to=2-1]
                \arrow["\alpha"', from=2-4, to=1-3]
                \arrow["\beta", from=2-4, to=3-3]
                \arrow["{g_2}", curve={height=-12pt}, from=3-3, to=2-1]
                \arrow["{g_1}", from=3-3, to=2-2]
            \end{tikzcd}\]
    \end{note}
\end{eg}

\begin{exercise}[3.9]
    \begin{itemize}
        \item \(\Obj(\sf{MSet})=\) 二元组 \((S, \sim )\), 其中 \(S\) 为集合, \(\sim \) 为等价关系.
        \item \(\Hom_{\sf{MSet}}((S, \sim ), (T, \approx)) = \) 从 \(S\) 到 \(T\) 的所有 functions \(f\), 使得 \(f\) 保持等价关系, 即
              \[
                  \forall s_1, s_2 \in S, s_1 \sim s_2 \implies f(s_1) \approx f(s_2).
              \]
    \end{itemize}
\end{exercise}
\begin{explanation}
    下面证明上述定义的 \(\sf{MSet}\) 是一个 category:
    \begin{itemize}
        \item 由于 functions 满足 associativity 与 composition law, 显然 \(\sf{MSet}\) 中的 morphisms 也满足 associativity 和 composition law.
        \item 对于任意 \((S, \sim )\), 存在一个 identity morphism \(\id_{(S, \sim )}\) 使得 \(\forall s \in S, \id_{(S, \sim )}(s) = s\).
        \item \(\id_{(S, \sim )}\) 显然也是 composition law 的 identity.
    \end{itemize}
\end{explanation}

\begin{exercise}[3.10]
    想要从 object \(A\) 中取出某个 subobject, 题干告诉我们这个过程是通过 morphisms \(A \to \Omega\) 来实现的, 容易想到特定的 subobject 由特定的 morphisms 指定, 那么 \(\Omega\) 的内容应该要能够指出哪些内容属于这个 subobject, 哪些不属于。

    在 \(\Set\) 中, \(\Omega\) 应该是一个集合, 简单起见, 我们可以直接令其为 \(\{0,1\}\), 其中 0 表示不属于 subobject, 1 表示属于 subobject.
\end{exercise}

\section{Morphisms}\label{sec:1.4}

在 \(\Set\) 中, 它的 morphisms 就是前面提到的 functions, 但是 morphisms 与 functions 有很大不同, morphisms 是一个更宽泛的概念, functions 的定义是 elementwise 的, 而 categoy 中的 objects 并不一定有 elements.

\subsection{Isomorphism}\label{sec:1.4.1}

\begin{definition}[Isomorphism]\label{def:isomorphism}
    A morphism \(f \in \Hom_{\Cat}(A, B)\) is an \emph{isomorphism} if it has a (two-sided) inverse under composition: if \(\exists g \in \Hom_{\Cat}(B, A)\) such that
    \[
        gf = 1_A, \quad fg = 1_B.
    \]
\end{definition}

由于 function 的定义是 elementwise 的, 所以一个 function 的 `inverse' 从定义上看天然是 unique 的, 而 isomorphism 的 `inverse' 则不然, 但它确实也是 unique 的, 只是需要一个额外的验证。

\begin{proposition}\label{prop:1.4.2}
    The inverse of an isomorphism is unique.
\end{proposition}
\begin{proof}
    若 \(g, h \in \Hom_{\Cat}(B, A)\) 都满足上式且 \(g \not= h\), 于是我们有
    \[g = g 1_{B} = g(fh) = (gf)h = 1_{A} h = h\]
    与假设矛盾。
\end{proof}

\begin{proposition}\label{prop:1.4.3}
    With notation as above:
    \begin{itemize}
        \item (reflective) Each identity \(1_A\) is an isomorphism and is its own inverse.
        \item (symmetric) If \(f\) is an isomorphism, then so is its inverse \(f^{-1}\) and further \((f^{-1})^{-1} = f\).
        \item (transitive) If \(f\in \Hom_{\Cat}(A, B), g \in \Hom_{\Cat}(B, A)\) are isomorphisms, then so is \(gf\) and \((gf)^{-1} = f^{-1}g^{-1}\).
    \end{itemize}
\end{proposition}
\begin{note}
    一个 category 中的两个 objects \(A, B\) 如果满足存在一个 isomorphism \(f \in \Hom_{\Cat}(A, B)\), 那么我们称 \(A\) 和 \(B\) 是 \emph{isomorphic}, 记作 \(A \cong B\).
\end{note}

\begin{eg}[Groupoid]\label{eg:1.4.6}
    如果一个 category 中所有的 morphisms 都是 isomorphism, 那么这个 category 被称为 \emph{groupoid}.
\end{eg}

一个 \emph{automorphism} 指的是 category \(\Cat\) 中的 object \(A\) 中所有在 \(\Hom_{\Cat}(A, A)\) 中的 isomorphisms 的集合, 记作 \(\Aut_{\Cat}(A)\), 它是 \(\End_{\Cat}(A)\) 的子集. 不难发现 \(\Aut_{\Cat}(A)\) 是一个 group.

\subsection{Monomorphisms and epimorphisms}\label{sec:1.4.2}

在 morphism 中类比于 injectivity 和 surjectivity 的概念分别是 \emph{monomorphism} 和 \emph{epimorphism}.

\begin{definition}[Monomorphism]\label{def:monomorphism}
    Let \(\Cat\) be a category. A morphism \(f \in \Hom_{\Cat}(A, B)\) is a \emph{monomorphism} if the following holds:
    \[
        \forall Z \in \Obj(\Cat), \forall g_1, g_2 \in \Hom_{\Cat}(Z, A), f \circ g_1 = f \circ g_2 \implies g_1 = g_2.
    \]
\end{definition}

\begin{definition}[Epimorphism]\label{def:epimorphism}
    Let \(\Cat\) be a category. A morphism \(f \in \Hom_{\Cat}(A, B)\) is an \emph{epimorphism} if the following holds:
    \[
        \forall Z \in \Obj(\Cat), \forall g_1, g_2 \in \Hom_{\Cat}(B, Z), g_1 \circ f = g_2 \circ f \implies g_1 = g_2.
    \]
\end{definition}

在 \(\Set\) 中,monomorphism 完全等价于 injectivity, epimorphism 完全等价于 surjectivity. 但是在由 \(\mathbb{Z}\) 上的关系 \(\le\) 定义得到的 category 中, 每个 morphism 都同时是 monomorphism 和 epimorphism, 但是唯一的 isomorphisms 只有 identies.

因此上述特性只是因为 \(\Set\) 是一个特殊的 category 而已, 更广泛来说, 上面这种特性会出现在所有 \emph{abelian category} 中.

\section{Universal properties}\label{sec:1.5}

从 \autoref{sec:1.3} 中的 examples 可以看出, 一个 category 可以通过一些操作来构造 (construction) 得到一些新的 categories, 而我们现在要研究的就是这些 constructions 之间的 \emph{universal properties}.

\subsection{Initial and final objects}\label{sec:1.5.1}

\begin{definition}[Initial and final objects]\label{def:initial_final_objects}
    Let \(\Cat\) be a category. We say that an object \(I\) of \(\Cat\) is \emph{initial} if for every object \(A\) of \(\Cat\), there exists a \emph{unique} morphism \(I \to A\) in \(\Cat\):
    \[
        \forall A \in \Obj(\Cat): \quad \Hom_{\Cat}(I, A) \text{ is a singleton.}
    \]
    We say that an object \(T\) of \(\Cat\) is \emph{final} if for every object \(A\) of \(\Cat\), there exists a \emph{unique} morphism \(A \to F\) in \(\Cat\):
    \[
        \forall A \in \Obj(\Cat): \quad \Hom_{\Cat}(A, F) \text{ is a singleton.}
    \]

    \begin{note}
        一个 category 不一定要有 initial 或者 final objects, 如果有的话, 它们不一定是 unique 的.
    \end{note}
\end{definition}

\begin{eg}[Set]\label{eg:1.5.3}
    在 \(\Set\) 中, 它的 initial object 是空集 \(\emptyset\), final objects 是任何 singleton.
\end{eg}

尽管 initial/final objects 不一定是 unique 的, 但是只要它们存在, 它们就一定是 isomorphic 的.

\begin{proposition}\label{prop:1.5.4}
    Let \(\Cat\) be a category.
    \begin{itemize}
        \item If \(I_1, I_2\) are both initial objects of \(\Cat\), then \(I_1 \cong I_2\).
        \item If \(F_1, F_2\) are both final objects of \(\Cat\), then \(F_1 \cong F_2\).
    \end{itemize}
    Further, these isomorphisms are unique determined.
\end{proposition}
\begin{proof}
    由 initial/final object 的定义, 任意两个 initial/final objects 之间各只会有一个 morphism, 因此这个 morphism 一定是 isomorphism.
\end{proof}

\subsection{Universal properties}\label{sec:1.5.2}

当某种 construction 可以看作某个 category 的 terminal object 时, 我们称它满足了某个 universal property. 换句话说, 一个 object 需要满足了某个 universal property 才能成为 terminal object.

一个简单的例子是我们称 \emph{\(\emptyset\) is universal with respect to the property of mapping to sets}, 这句话和 ``\(\emptyset\) is the initial object in \(\Set\)'' 是等价的.

在更多的时候 initial/final objects 对应的 properties 会更复杂, 对于一个 universal property 的 `explanation' 可能会遵循以下模式: ``object \(X\) is universal with respect to the following property: for any \(Y\) such that\dots, there exists a unique morphism \(Y \to X\) such that\dots''.

\subsection{Quotients}\label{sec:1.5.3}

设 \(\sim \) 是定义在集合 \(A\) 上的一个等价关系, 我们有:

\emph{``The quotient \(A/\sim \) is universal with respect to the property of mapping \(A\) to a set in such a way that equivalent elements have the same image.''}

\subsection{Products}\label{sec:1.5.4}

假设 \(A, B\) 是两个集合, 考虑它们的 product \(A \times B\) 以及两个 natural projections:

% https://q.uiver.app/#q=WzAsMyxbMCwxLCJBIFxcdGltZXMgQiJdLFsxLDAsIkEiXSxbMSwyLCJCIl0sWzAsMSwiXFxwaV9BIl0sWzAsMiwiXFxwaV9CIiwyXV0=
\[\begin{tikzcd}
        & A \\
        {A \times B} \\
        & B
        \arrow["{\pi_A}", from=2-1, to=1-2]
        \arrow["{\pi_B}"', from=2-1, to=3-2]
    \end{tikzcd}\]

我们有, 对任意集合 \(Z\) 和 morphisms

% https://q.uiver.app/#q=WzAsMyxbMCwxLCJaIl0sWzEsMCwiQSJdLFsxLDIsIkIiXSxbMCwxLCJmX0EiXSxbMCwyLCJmX0IiLDJdXQ==
\[\begin{tikzcd}
        & A \\
        Z \\
        & B
        \arrow["{f_A}", from=2-1, to=1-2]
        \arrow["{f_B}"', from=2-1, to=3-2]
    \end{tikzcd}\]

存在一个 unique morphism \(\sigma \colon Z \to A \times B\) 使得 diagrams

% https://q.uiver.app/#q=WzAsNCxbMCwxLCJaIl0sWzEsMSwiQSBcXHRpbWVzIEIiXSxbMiwwLCJBIl0sWzIsMiwiQiJdLFsxLDIsIlxccGlfQSJdLFsxLDMsIlxccGlfQiIsMl0sWzAsMSwiXFxzaWdtYSJdLFswLDIsImZfQSIsMCx7ImN1cnZlIjotMn1dLFswLDMsImZfQiIsMix7ImN1cnZlIjoyfV1d
\[\begin{tikzcd}
        && A \\
        Z & {A \times B} \\
        && B
        \arrow["{f_A}", curve={height=-12pt}, from=2-1, to=1-3]
        \arrow["\sigma", from=2-1, to=2-2]
        \arrow["{f_B}"', curve={height=12pt}, from=2-1, to=3-3]
        \arrow["{\pi_A}", from=2-2, to=1-3]
        \arrow["{\pi_B}"', from=2-2, to=3-3]
    \end{tikzcd}\]

commutes.

此时, \(\sigma\) 通常记为 \(f_A \times f_B\).

\begin{proof}
    \(\forall z \in Z\), 令
    \[
        \sigma(z) = (f_A(z), f_B(z)).
    \]
\end{proof}

不难发现, \(A \times B\) (together with the information of their natural projections to the factors) 是 \(\Cat_{A, B}\) 中的 final object.

当 category \(\Cat\) 中的任意两个 objects \(A, B\) 在 \(\Cat_{A, B}\) 中都有 final objects 时, 我们称 \(\Cat\) 有 \emph{(finite) products}.

通过这种视角来看待 products 的主要好处就是, universal property 可能在任意 category 中成立, 而不仅仅是在 \(\Set\) 中.

\subsection{Coproducts}\label{sec:1.5.5}

类似地, 给定两个 morphisms \(i_A : A \to A \coprod B, i_B : B \to A \coprod B\), 我们有 \emph{coproduct} \(A \coprod B\) 是 \(\Cat^{A, B}\) 中的 initial object, 满足下述 universal property: 对任意 object \(Z\) 和 morphisms

% https://q.uiver.app/#q=WzAsMyxbMCwwLCJBIl0sWzEsMSwiWiJdLFswLDIsIkIiXSxbMCwxLCJmX0EiXSxbMiwxLCJmX0IiLDJdXQ==
\[\begin{tikzcd}
        A \\
        & Z \\
        B
        \arrow["{f_A}", from=1-1, to=2-2]
        \arrow["{f_B}"', from=3-1, to=2-2]
    \end{tikzcd}\]

存在一个 unique morphism \(\sigma : A \coprod B \to Z\) 使得 diagrams

% https://q.uiver.app/#q=WzAsNCxbMCwwLCJBIl0sWzAsMiwiQiJdLFsxLDEsIkEgXFxjb3Byb2QgQiJdLFsyLDEsIloiXSxbMCwyLCJpX0EiXSxbMSwyLCJpX0IiLDJdLFsyLDMsIlxcc2lnbWEiXSxbMCwzLCJmX0EiLDAseyJjdXJ2ZSI6LTJ9XSxbMSwzLCJmX0IiLDAseyJjdXJ2ZSI6Mn1dXQ==
\[\begin{tikzcd}
        A \\
        & {A \coprod B} & Z \\
        B
        \arrow["{i_A}", from=1-1, to=2-2]
        \arrow["{f_A}", curve={height=-12pt}, from=1-1, to=2-3]
        \arrow["\sigma", from=2-2, to=2-3]
        \arrow["{i_B}"', from=3-1, to=2-2]
        \arrow["{f_B}", curve={height=12pt}, from=3-1, to=2-3]
    \end{tikzcd}\]

commutes.

\begin{proposition}[Coproducts in \(\Set\)]\label{prop:1.5.6}
    The disjoint union \emph{is a coproduct in \(\Set\)}.
\end{proposition}