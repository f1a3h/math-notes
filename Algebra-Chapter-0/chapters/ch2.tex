\chapter{Groups, first encounter}

前面学习了 category 的概念,现在我们学习一种特殊的 category,叫做 group(称为 \(\Grp\)).

\section{Definition of group}\label{sec:2.1}

\subsection{Groups and groupoids}\label{sec:2.1.1}

\begin{quote}[Joke]
    \emph{Definition}: A group \emph{is a groupoid with a single object}.
\end{quote}

这个定义很好的体现了 groups 的 properties,在 groupoid \(\mathsf{G}\) 中,所有 morphisms 都是 isomorphism,当 object 只有一个(记为 \(*\))时,那么这些 morphisms 都是 automorphism,即

\[\Hom_{\sf{G}}(*, *) = \Aut_{\sf{G}}(*)\]

将 set \(\Hom_{\sf{G}}(*, *)\) 记作 \(G\),可以发现 \(G\) 对 composition 是封闭的、每个 morphism 都有 inverse、有单位元 \(1_{*}\),事实上,\(G\) 加上 composition law 才是一个 group,前面的定义错在将 group 当成了一个 groupoid with a single object.

\subsection{Definition}\label{sec:2.1.2}

在正式的定义中,将上述 composition law 记作 \(\bullet\),称为 `mulplication',notation 形如:

\[\bullet \colon G \times G \to G\]

或

\[\bullet(g, h) \eqqcolon g \bullet h\]

\begin{definition}[Group]\label{def:group}
    Let \(G\) be a nonempty set, endowed with the binary operation \(\bullet\) (briefly, \((G, \bullet)\) or simply \(G\)), is a \emph{group} if
    \begin{enumerate}
        \item the operation \(\bullet\) is \emph{associative};
        \item there exists an \emph{identity element} \(e_{G}\) for \(\bullet\);
        \item every element in \(G\) has an \emph{inverse} with respect to \(\bullet\).
    \end{enumerate}
\end{definition}

最简单的 group 显然是只包含 \(e_{G}\) 一个元素的 group,称为 \emph{trivial group}.

\begin{eg}\label{eg:2.1.5}
    可逆的 \(n \times n\) matrices with real entries 的集合与矩阵乘法运算可以构成一个 group,表示为 \(\GL_{n}(\mathbb{R})\).
\end{eg}

\subsection{Basic properties}\label{sec:2.1.3}

从前面 groupoid 的角度来看,\(G\) 中的 \(e_G\) 显然对应的是 \(1_{*}\),但是正式的定义里并没有说明这一点,甚至没有要求 \(e_G\) 是 unique 的,实际上,\(G\) 的元素中除了 \(1_{*}\) 没有其他的元素能 work as an identity.

\begin{proposition}\label{prop:2.1.6}
    If \(h \in G\) is an identity of \(G\), then \(h = e_G\).
\end{proposition}
\begin{explanation}
    直接使用 identity element 的定义导出矛盾。假设 \(h\) 是一个 identity 且 \(h \not= e_G\),那么有
    \[h = e_{G} h = e_{G}\]
\end{explanation}

\begin{proposition}\label{prop:2.1.7}
    The inverse is also unique.
\end{proposition}
\begin{explanation}
    由于 \(G\) 中的每个元素都是 isomorphism,因此根据 \autoref{prop:1.4.2} 这个引理显然是成立的。
\end{explanation}

\autoref{prop:2.1.7} 启示我们可以将元素 \(g\) 的 inverse 记作 \(g^{-1}\),进一步地,我们将一个元素对自身做 \(n\) 次的 multiplication 记作 \(g^n\),它的 inverse 记作 \(g^{-n}\),于是我们可以很自然地定义 \(g^0 = e_G\).

\subsection{Cancellation}\label{sec:2.1.4}

`cancellation' 在 group 中同样存在,由于 isomorphism 既是 monomorphism 又是 epimorphism,因此 \(G\) 中的每个元素都是左右可消的。

\begin{note}
    反过来不成立,如果一个 morphism 即是 monomorphism 又是 epimorphism,那么它也不一定是 isomorphism,由偏序关系 \(<\) 导出的 category 就是一个反例。
\end{note}

\begin{proposition}\label{prop:2.1.8}
    Let \(G\) be a group. Then \(\forall a, g, h \in G\)
    \[ga = ha \implies g = h, \quad ag = ah \implies g = h.\]
\end{proposition}
\begin{explanation}
    因为 group 中每个元素都有 inverse,因此等式两边同时乘上 \(a^{-1}\) 即可得证。
\end{explanation}

\subsection{Commutative groups}\label{sec:2.1.5}

我们称对 \(\bullet\) 满足 commutative law 的 group 为 commutative group 或 \emph{abelian group}.

abelian groups 有一个很重要的性质,那就是它的结构是 `\(\mathbb{Z}\)-module structure',这种性质在当 abelian groups 与其他 operations 同时存在的当情况下很有用。

换句话说,abelian groups 和 \(\mathbb{Z}\)-module 是同构的,那么我们可以直接将它作为 \(\mathbb{Z}\)-module 来处理。

注意到 \(\mathbb{Z}\)-module 中的 \(\bullet\) 对应的是 \(+\),因此 abelian groups \(A\) 中的 \(\bullet\) 可以用 \(+\) 来表示,\(0_A\) 即是 \emph{identity},\(a \in A\) 的 inverse 则是 \(-a\),前面使用的 `power' notation 则成为了 `multiple': \(0a = 0\),对于正整数 \(n\),有
\[na = \underbrace{a + \cdots + a}_{n \text{ times}}, \quad (-n)a = \underbrace{(-a) + \cdots + (-a)}_{n \text{ times}}.\]

\begin{remark}
    这里的 \(na\) 只是一种 notation,并不存在 `cancellation'.
\end{remark}